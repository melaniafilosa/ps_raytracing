\chapter*{Summary}
%\pagestyle{empty}
%\pagenumbering{}
\markboth{}{}
\addcontentsline{toc}{chapter}{Summary}
In this thesis we studied the light propagation within optical systems.
Optical engineers are interested in design systems in such a way the desired output distribution of light is obtained. Ray tracing is a forward method where a set of rays is traced within the system from the source to the target. The propagation of light is determined computing the position and the direction of every ray for all the optical surfaces that it encounters. There are many ways to implement the ray tracing process. Monte Carlo (MC) ray tracing is often used in non-imaging optics. Rays are randomly traced from the source to the target and each time that a ray hits an optical surface the coordinates of the intersection point of the ray with the surface and the new ray direction are calculated. The output variables are computed dividing the target into intervals, the so-called bins, and counting the rays that fall into each bin. To obtain the desired accuracy, millions of rays are required, therefore the method is extremely computationally expensive and it converges as the inverse of the square root of the number of rays traced. 
\\ \indent 
MC ray tracing can be improved using as sample of points a low discrepancy sequence instead of random points. Discrepancy can be interpreted as a measure of how much the sample distribution differs from a uniformly distributed sample. The discrepancy is therefore zero for uniformly distributed points. A low discrepancy sequence gives a sample of points which are regularly distributed but not exactly uniformly distributed. Quasi Monte Carlo (QMC) ray tracing is implemented tracing a set of rays whose position and direction are given by the coordinates of a low discrepancy sequence of points.
The main advantage of QMC ray tracing is its rate of convergence, it is faster then MC for low dimensional problems. Nevertheless, it has some disadvantages. First, it is not easy to give an error estimation for QMC ray tracing. Second, for high dimensional spaces the QMC can become very slow. Third, it is still a binning procedure. Hence, the accuracy depends both on the number of rays traced and on the number of bins.
\\ \indent
In order to improve the existing methods, the phase space (PS) of the optical system is considered in this thesis. The PS of an optical surface gives information about the position and the direction of every ray on that surface where the direction is expressed with respect to the normal of the surface. In PS, the ray's direction is given by the sine of the angle that the ray forms with respect to the normal of the surface multiplied by the index of refraction of the medium in which the ray is located.
In two dimensions, the PS is a two-dimensional space where the coordinates of every ray are specified by one position coordinate and one angular coordinate. 
For three dimensional systems the PS is a four dimensional space because every ray is specified by two position and two angular coordinates. Our idea is to use the structure of PS to trace only the rays close to the discontinuities of the luminance at the target PS.
Two new approaches based on PS are presented in this work. They are tested for two-dimensional systems. 
\\ \indent The first method is called ray tracing on PS and it is based on the source and the target PS representation of the optical system. It takes into account the sequence of optical lines that each ray hits when it propagates inside the system, that is the ray path. We note that the source and target phase spaces are partitioned into different regions each of them is formed by the rays that follow the same path. The idea is to use the edge-ray principle proved by Ries and Rabl (1994) which states that the area of these regions is conserved: all rays that are neighbors at the source PS remain close to each other at the target PS. To this purpose, a nonuniform triangulation of the source PS is constructed in such a way that new triangles are added to the triangulation only where boundaries occur. 
Assuming constant brightness, we only need to compute the boundaries of the regions in target PS to obtain the output photometric variables. 
We test the method for optical systems where both reflection and refraction laws are involved.
 Numerical results show that ray tracing on PS is faster and more accurate compared to MC ray tracing.
% Nevertheless, PS ray tracing can be further improved. Indeed, the triangulation refinement allows tracing only the rays close to the boundaries and an approximation of those boundaries is needed to compute the output intensity distribution. The ideal situation would be that only the rays located exactly on the boundaries of the regions at target PS are traced inside the system
\\ \indent The second method is called backward ray mapping, it employs not only the source and the target PS, but also the PS of \textit{all} the other lines that constitute the system.
All lines can be modeled as detectors of the incident light and emitters of the reflected light.
Moreover, we assume that the source can only emit light and the target can only receive light.
Therefore, one PS is taken into account for the source and one for the target. For the other surfaces both the source and target PS are considered.
Furthermore, instead of starting from the source, the new method starts tracing back rays from target PS. 
In order to determine the coordinates of these rays, an inverse map from the target to the source PS is constructed as a concatenation of the maps that relate the PS of two different lines.
Employing this map we are able to detect the rays that in target PS are located on the boundaries of the regions with positive luminance. 
First, we implement the method for systems formed by straight and reflective lines. In this particular case, the boundaries of the regions that form every PS can be computed analytically. This allows us to obtain an analytic target intensity distribution. The results are shown for a two-faceted cup and a multi-faceted cup. In both cases we note significant advantages both in terms of the accuracy and the computational time. Second, the method is developed for systems formed by curved lines. In this case the boundaries cannot be determined analytically and therefore a numerical procedure is involved. In particular, we apply a bisection method on target PS. Also in this case we compare our method to MC ray tracing and we observe significant advantages using the PS method. Finally, the ray mapping method in PS is applied to systems where also Fresnel reflection is taken into account. We obtain relevant results also in the last case.