\chapter{Two examples of low discrepancy sequences}\label{app:Sobol}
A lot of researchers have concentrated on the study of low discrepancy sequences widely used in QMC method. We here provide some details on two types of those sequences: The Van der Corput sequence and the Sobol sequence.
In the following we give a brief outline of the theory reported in \cite{joe2008notes, leobacher2014introduction}.
\section{Van der Corput sequences}
\textit{Van der Corput} sequence takes the name by its inventor Van der Corput who introduced it the first time in 1935. In the following we show how to construct such sequence in one dimension $d=1$. This kind of sequences are particularly relevant because many other kind of sequences in higher dimensions are based on this one-dimensional case. Before going in details we give the concept of radical inverse function. \\ \indent 
Let $\variabile{b}\geq 2$ be an integer number. Any natural number $\variabile{n}\in \mathbb{N}_0$ can be decomposed in base $\variabile{b}$ as follows:
\begin{equation}
\variabile{n} = \sum_{\variabile{i}=0}^\infty \variabile{d}_{\variabile{i}}\variabile{b}^{\variabile{i}}
\end{equation}
where $\variabile{d}_{\variabile{i}} \in \{0, 1, \cdots, \variabile{b}-1\}$ are called the digit numbers.
\begin{definition}
The \textit{radical inverse function} $\phi_{\variabile{b}}:\mathbb{N}_0\mapsto [0,1)$ in base $\variabile{b}$ is defined as:
\begin{equation}
\phi_{\variabile{b}}(\variabile{n}) = \sum_{\variabile{i}=1}^{\infty}\frac{\variabile{d}_{\variabile{i}-1}}{\variabile{b}^{\variabile{i}}}.
\end{equation}
\end{definition}
As an example we calculate the radical inverse function $\phi_{\variabile{b}}(5)$ in base $\variabile{b} = 2$. 
The digit expansion in base $\variabile{b}$ of $\variabile{n}=5$ is:
\begin{equation}
5 = 1\cdot 2^0+1\cdot 2^2.
\end{equation}
Therefore, $\variabile{d}_0 = 1, \variabile{d}_1 = 0$ and $\variabile{d}_2 = 1$. 
The radical inverse function $\phi_2(5)$ is:
\begin{equation}
\phi_2 (5) = \frac{1}{2}+\frac{1}{8} = \frac{5}{8}.
\end{equation}
\begin{definition}
The Van der Corput sequence in base $\variabile{b}$ is defined as $\{ \phi_{\variabile{b}}(\variabile{n})\}_{n\in\mathbb{N}_0}$.
\end{definition}
For example, suppose we have the finite sequence of numbers $\variabile{n}\in \{0, 1,\cdots, 8\}$  the corresponding Van der Corput sequence 
$\{ \phi_{\variabile{b}}(\variabile{n})\}_{\variabile{n}\in \{0, 1,\cdots, 8\}}$ in base $\variabile{b}=2$ is:
\begin{equation}
\big\{\phi_2(\variabile{n})\big\}_{\variabile{n}\in \{0, 1,\cdots, 8\}} = \Bigg\{0, \frac{1}{2}, \frac{1}{4}, \frac{3}{4}, \frac{1}{8},\frac{5}{8}, \frac{3}{8}, \frac{7}{8}, \frac{1}{16}\Bigg\} \,.
\end{equation}
It can be proved that the Van der Corput sequence in base $\variabile{b}$ is uniformly distributed modulo one \cite{leobacher2014introduction}. 
The van der Corput sequence has been extended to higher dimensions. 
Sobol sequence which can be seen as an extended Van der Corput sequence in base $\variabile{b}=2$ for every dimension $\variabile{d}\geq2$. 
\section{Sobol sequences}
The aim is to generate a low-discrepancy sequence in the ipercube $[0,1]^{\variabile{d}}$. 
Let us start from the simplest case of one dimension, i.e. $d=1$. First, we need to chose a primitive polynomial $P_\variabile{j}$ of degree $\textrm{s}_\variabile{j}$ of the form
\begin{equation}
P_{\variabile{j}}:\variabile{x}^{\textrm{s}_{\variabile{j}}}+\variabile{a}_{1, \variabile{j}}\variabile{x}^{\textrm{s}_{\variabile{j}}-1}+\cdots+\variabile{a}_{\textrm{s}_{\variabile{j}}-1}\variabile{x}+1
\end{equation}
where the coefficients $\{\variabile{a}_{\variabile{i},\variabile{j}}\}_{\variabile{i}= 1, \cdots, \textrm{s}_{\variabile{j}}-1}$ are either $0$ or $1$ \cite{joe2008constructing}. 
Then a sequence $\{m_1,m_2,\cdots\}$ is defined such that:
\begin{equation}\label{eq:m}
m_{\variabile{k}, \variabile{j}} := 2\variabile{a}_{1, \variabile{j}}m_{\variabile{k}-1, \variabile{j}}\oplus 2^2a_{2, \variabile{j}}m_{\variabile{k}-2, \variabile{j}}\oplus \cdots \oplus 2^{\textrm{s}-1}\variabile{a}_{\variabile{k}-1, \variabile{j}}m_{\variabile{k}-\textrm{s}+1, \variabile{j}}\oplus 2^{\textrm{s}}m_{\variabile{k}-1, \variabile{j}}\oplus m_{\variabile{k}-\textrm{s}, \variabile{j}},
\end{equation}
where we have indicated with $\oplus$ the bit by bit exclusive or operator which operates on two bit patterns and operates on each pair of the corresponding bins giving as result $1$ if one of the two bits is $1$ and $0$ if both bits are equal either to $0$ or $1$. The values $m_{\variabile{k}, \variabile{j}}$, $1\leq\variabile{k}\leq d$, are chosen such that they are odd and positive numbers less than $2^\variabile{k}$. Now, the so-called direction numbers are defined by:
\begin{equation}\label{eq:dir_numb}
\variabile{v}_{\variabile{k}, \variabile{j}}=\frac{m_{\variabile{k}, \variabile{j}}}{2^\variabile{k}}.
\end{equation}
Then, the sequence $\{\variabile{x}_{\variabile{i}, \variabile{j}}\}$ is given by
\begin{equation}\label{eq:x}
\variabile{x}_{\variabile{i}, \variabile{j}} = \variabile{i}_1\variabile{v}_1\oplus \variabile{i}_2\variabile{v}_{2}\oplus \cdots
\end{equation}
for every $\variabile{i}$, where $\variabile{i}_{\variabile{k}}$ is the $\variabile{k}$-th digit from the right when $\variabile{i}$ is written in binary $\variabile{i} = (\cdots \variabile{i}_3\variabile{i}_2\variabile{i}_1)_2$, \cite{joe2008notes}. We provide in the following an example. \\ \indent
Given the primitive polynomial $\variabile{x}^3+\variabile{x}^2+1$ of degree $\textrm{s}_{\variabile{j}}=3$, the first three coefficients $m_{1, \variabile{j}}=1$, 
$m_{2, \variabile{j}}=3$, and $m_{3, \variabile{j}}=7$ lead to the following direction numbers 
\begin{equation}
\variabile{v}_{1,\variabile{j}} = \frac{1}{2}, \qquad \variabile{v}_{2,\variabile{j}} = \frac{3}{4}, \qquad \variabile{v}_{3,\variabile{j}}= \frac{7}{8},
\end{equation}
that in binary notation are:
\begin{equation}
\variabile{v}_{1,\variabile{j}} = (0.1)_2 \qquad \variabile{v}_{1,\variabile{j}} = (0.11)_2, \qquad \variabile{v}_{1,\variabile{j}} = (0.111)_2.
\end{equation}
From Eq. (\ref{eq:m}) we can derive the others coefficients $m_{4,\variabile{j}}=5$, $m_{5,\variabile{j}}=7$, etc. with he corresponding direction vectors:
\begin{equation}
\variabile{v}_{4,\variabile{j}} = \frac{5}{16} = (0.0101)_2  \qquad \variabile{v}_{5,\variabile{j}} = \frac{7}{32}=(0.00111)_2
\end{equation}
From Eq. (\ref{eq:x}) we finally find the sequence
\begin{equation}
\end{equation}
\\ \indent
The generalization of Sobol sequence to higher dimensions $d>1$ is calculated considering a sequence where the $\variabile{i}$-th point has the form:
\begin{equation}
\variabile{q}_{i} = (\variabile{x}_{\variabile{i}, 1},\variabile{x}_{\variabile{i}, 2} \cdots, \variabile{x}_{\variabile{i}, d}),
\end{equation}
where the second index of the variables $\variabile{x}_{\variabile{i}, \variabile{j}}$ it refers to the polynomial $P_{\variabile{j}}$ (with corresponding degree $\textrm{s}_{\variabile{j}}$) which is considered to calculate the direction numbers. Therefore, $d$ different sets of direction numbers are generated from a given polynomial $P_\variabile{j}$ using Eq. (\ref{eq:dir_numb}) and each component $\variabile{x}_{\variabile{i},\variabile{j}}$ is computed using the corresponding direction vector. 